\documentclass[journal,12pt,twocolumn]{IEEEtran}
%
\usepackage{setspace}
\usepackage{gensymb}
\usepackage{xcolor}
\usepackage{caption}
%\usepackage{subcaption}
%\doublespacing
\singlespacing

%\usepackage{graphicx}
%\usepackage{amssymb}
%\usepackage{relsize}
\usepackage[cmex10]{amsmath}
\usepackage{mathtools}
%\usepackage{amsthm}
%\interdisplaylinepenalty=2500
%\savesymbol{iint}
%\usepackage{txfonts}
%\restoresymbol{TXF}{iint}
%\usepackage{wasysym}
\usepackage{amsthm}
\usepackage{mathrsfs}
\usepackage{txfonts}
\usepackage{stfloats}
\usepackage{cite}
\usepackage{cases}
\usepackage{subfig}
%\usepackage{xtab}
\usepackage{longtable}
\usepackage{multirow}
%\usepackage{algorithm}
%\usepackage{algpseudocode}
\usepackage{enumitem}
\usepackage{mathtools}
%\usepackage{eenrc}
%\usepackage[framemethod=tikz]{mdframed}
\usepackage{hyperref}
\usepackage{listings}
\usepackage[latin1]{inputenc}                                 %%
\usepackage{color}                                            %%
\usepackage{array}                                            %%
\usepackage{longtable}                                        %%
\usepackage{calc}                                             %%
\usepackage{multirow}                                         %%
\usepackage{hhline}                                           %%
\usepackage{ifthen}                                           %%
%optionally (for landscape tables embedded in another document): %%
\usepackage{lscape}     
\usepackage{tikz}
\usepackage{circuitikz}
\usepackage{karnaugh-map}
\usepackage{pgf}

\usepackage{url}
\def\UrlBreaks{\do\/\do-}



%\usepackage{stmaryrd}


%\usepackage{wasysym}
%\newcounter{MYtempeqncnt}
\DeclareMathOperator*{\Res}{Res}
%\renewcommand{\baselinestretch}{2}
\renewcommand\thesection{\arabic{section}}
\renewcommand\thesubsection{\thesection.\arabic{subsection}}
\renewcommand\thesubsubsection{\thesubsection.\arabic{subsubsection}}

\renewcommand\thesectiondis{\arabic{section}}
\renewcommand\thesubsectiondis{\thesectiondis.\arabic{subsection}}
\renewcommand\thesubsubsectiondis{\thesubsectiondis.\arabic{subsubsection}}

% correct bad hyphenation here
\hyphenation{op-tical net-works semi-conduc-tor}

%\lstset{
	%language=C,
	%frame=single, 
	%breaklines=true
	%}

%\lstset{
	%%basicstyle=\small\ttfamily\bfseries,
	%%numberstyle=\small\ttfamily,
	%language=Octave,
	%backgroundcolor=\color{white},
	%%frame=single,
	%%keywordstyle=\bfseries,
	%%breaklines=true,
	%%showstringspaces=false,
	%%xleftmargin=-10mm,
	%%aboveskip=-1mm,
	%%belowskip=0mm
	%}

%\surroundwithmdframed[width=\columnwidth]{lstlisting}
\def\inputGnumericTable{}                                 %%
\lstset{
	%language=C,
	frame=single, 
	breaklines=true,
	columns=fullflexible
}


\graphicspath{ {./title/} } 
\begin{document}
	
	
	\title{
		
		
		\includegraphics[width=15cm, height=4cm]{title}
		\centering
		
	}
	\author{G V V Sharma$^{*}$% <-this % stops a space
		\thanks{*The author is with the Department
			of Electrical Engineering, Indian Institute of Technology, Hyderabad
			502285 India e-mail:  gadepall@iith.ac.in. All content in this manual is released under GNU GPL.  Free and open source.}}
	
	\maketitle
	\tableofcontents


\bigskip
%
\begin{abstract}
This manual shows how to use the 7447 BCD-Seven Segment Display decoder to learn Boolean logic.
\end{abstract}
%\newpage
\section{Components}

	\begin{table}[h!]
	%	\begin{adjustbox}{width=\columnwidth,center}	
		\begin{center}
			\begin{tabular}{ |c|c|c| } 
				\hline
				\textbf{Component} & \textbf{Value} & \textbf{Quantity} \\ 
				\hline
				Breadboard &  & 1  \\ 
				\hline
				Resistor &  220 $\Omega$ & 1  \\ 
				\hline
				Arduino & Uno & 1  \\ 
				\hline
				Seven Segment & Common & 1 \\
				Display &  Anode &   \\
				\hline
			    Decoder & 7447 & 1  \\
				\hline
				Jumper Wires & M-M & 20  \\
				\hline
			\end{tabular}
			\caption{}
			\label{table:components}
		\end{center}
		%\end{adjustbox}
	\end{table}
\section{Hardware}
%\begin{problem}
\textbf{Problem 2.1.} Make connections between the seven segment display in Fig. \ref{fig:sevenseg} and the  7447 IC in Fig. \ref{fig:7447} as shown in Table \ref{table:7447_disp} \\
%\end{problem}
%
\begin{table}[!h]
\centering
\input{./figs/7447_disp.tex}
\caption{}
\label{table:7447_disp}
\end{table}
\begin{figure}[!h]
\begin{center}
	\includegraphics[width=6cm,height=9cm]{./sevenseg}
\end{center}
\caption{}
\label{fig:sevenseg}
\end{figure}
%\begin{problem}
\textbf{Problem 2.2.} Make connections to the lower pins of the 7447 according to
Table \ref{table:bin2dec} and connect $V_{CC} = 5$V. You should see the number 0 displayed for 0000 and 1 for 0001.\\
%\end{problem}
%
\begin{table}[!h]
\centering
\input{./figs/bin2dec.tex}
\caption{}
\label{table:bin2dec}
\end{table}
%
\begin{figure}[!h]
\begin{center}
	\resizebox {\columnwidth} {!} {
		\input{./figs/7447.tex}
	}
\end{center}
\caption{}
\label{fig:7447}
\end{figure}
%
%\begin{problem}
\textbf{Problem 2.3.} Complete Table \ref{table:bin2dec} by generating all numbers between 0-9.
%\end{problem}
\section{Software}
%\begin{problem}
\textbf{Problem 3.1.} Now make the connections as per Table \ref{table:7447_ard}  and execute the following program after downloading
\begin{lstlisting}
	wget https://raw.githubusercontent.com/gadepall/arduino/master/7447/codes/gvv_ard_7447/gvv_ard_7447.ino
\end{lstlisting}
%\end{problem}
\begin{table}[!h]
\centering
\input{./figs/7447_ard.tex}
\caption{}
\label{table:7447_ard}
\end{table}
In the  truth table in Table \ref{table:counter_decoder},  $W,X,Y,Z$ are the inputs
and $A,B,C,D$ are the outputs. This table represents the system that increments the numbers 0-8 by 1 and resets the number 9 to 0
%
Note that  $D = 1$ for the inputs $0111$ and $1000$.  Using {\em boolean} logic,
%
\begin{equation}
\label{bool_logic}
D = WXYZ^{'} + W^{'}X^{'}Y^{'}Z
\end{equation}
%
Note that $0111$ results in the expression $WXYZ^{'}$ and $1000$ yields $W^{'}X^{'}Y^{'}Z$. \\
%
%\begin{problem}
\textbf{Problem 3.2.} The code below realizes the Boolean logic for B, C and D in  Table \ref{table:counter_decoder}.  Write the logic for A and verify.
\begin{lstlisting}
	wget https://raw.githubusercontent.com/gadepall/arduino/master/7447/codes/inc_dec/inc_dec.ino
\end{lstlisting}
%\end{problem}
\begin{table}[!h]
\centering	
\input{./figs/counter_decoder.tex}
\caption{}
\label{table:counter_decoder}
\end{table}
%\begin{problem}
\textbf{Problem 3.3.} Now make additional connections as shown in Table \ref{table:ip_7447_ard} and execute the following code.  Comment.
\begin{lstlisting}
	wget https://raw.githubusercontent.com/gadepall/arduino/master/7447/codes/ip_inc_dec/ip_inc_dec.ino
\end{lstlisting}
%\end{problem}
%\solution
\textbf{Solution:} In this exercise, we are taking the number 5 as input to the arduino and displaying it on the seven segment display using the 7447 IC.\\
\begin{table}[!h]
\begin{center}

\input{./figs/ip_7447_ard.tex}
\caption{}
\label{table:ip_7447_ard}

\end{center}
\end{table}
%\begin{problem}
\textbf{Problem 3.4.} Verify the above code for all inputs from 0-9.\\
%\end{problem}
%\begin{problem}
\textbf{Problem 3.5.} Now write a program where 
\begin{enumerate}
	\item the binary inputs are given by
	connecting to 0 and 1 on the breadboard
	\item incremented by 1 using Table \ref{table:counter_decoder} and
	\item the incremented value is displayed on the seven segment display.
\end{enumerate}
%\end{problem}
%\begin{problem}
\textbf{Problem 3.6.} Write the truth table for the 7447 IC and obtain the corresponding boolean logic equations. \\
%\end{problem}
%\begin{problem}
\textbf{Problem 3.7.} Implement the 7447 logic in the arudino.  Verify that your arduino now behaves like the 7447 IC.
%\end{problem}
\end{document}

