\documentclass[journal,12pt,twocolumn]{IEEEtran}
\usepackage{graphicx}
\usepackage{listings}
\usepackage{setspace}
\usepackage{gensymb}
\usepackage{xcolor}
\usepackage{caption}
%\usepackage{subcaption}
%\doublespacing
\singlespacing

%\usepackage{graphicx}
%\usepackage{amssymb}
%\usepackage{relsize}
\usepackage[cmex10]{amsmath}
\usepackage{mathtools}
%\usepackage{amsthm}
%\interdisplaylinepenalty=2500
%\savesymbol{iint}
%\usepackage{txfonts}
%\restoresymbol{TXF}{iint}
%\usepackage{wasysym}
\usepackage{amsthm}
\usepackage{mathrsfs}
\usepackage{txfonts}
\usepackage{stfloats}
\usepackage{cite}
\usepackage{cases}
\usepackage{subfig}
%\usepackage{xtab}
\usepackage{longtable}
\usepackage{multirow}
%\usepackage{algorithm}
%\usepackage{algpseudocode}
\usepackage{enumitem}
\usepackage{mathtools}
%\usepackage{eenrc}
%\usepackage[framemethod=tikz]{mdframed}
\usepackage{listings}
\usepackage{listings}
\usepackage[latin1]{inputenc}                                 %%
\usepackage{color}                                            %%
\usepackage{array}                                            %%
\usepackage{longtable}                                        %%
\usepackage{calc}                                             %%
\usepackage{multirow}                                         %%
\usepackage{hhline}                                           %%
\usepackage{ifthen}                                           %%
%optionally (for landscape tables embedded in another document): %%
\usepackage{lscape}     
\usepackage{adjustbox}
\usepackage{tikz}
\graphicspath{ {./title/} } 
\begin{document}


	\title{
		
	
		\includegraphics[width=15cm, height=4cm]{title}
		\centering

		 }
	\author{G V V Sharma$^{*}$% <-this % stops a space
		\thanks{*The author is with the Department
			of Electrical Engineering, Indian Institute of Technology, Hyderabad
			502285 India e-mail:  gadepall@iith.ac.in. All content in this manual is released under GNU GPL.  Free and open source.}}
	
	\maketitle
	\tableofcontents

	\begin{abstract}
		The objective of this manual is to show how to control
		a seven segment display through the arduino IDE.
	\end{abstract}
	\section{Components}
	%\input{./figs/components.tex}
		\begin{table}[h!]
	%	\begin{adjustbox}{width=\columnwidth,center}	
	\begin{center}
			\begin{tabular}{ |c|c|c| } 
				\hline
				\textbf{Component} & \textbf{Value} & \textbf{Quantity} \\ 
				\hline
				Breadboard &  & 1  \\ 
				\hline
				Resistor & $\ge 220 \Omega$ & 1  \\ 
				\hline
				Arduino & Uno & 1  \\ 
				\hline
				Seven Segment & Common & 1 \\
				Display &  Anode &   \\
				\hline
				Jumper Wires &  & 20  \\
				\hline
			\end{tabular}
			\caption{}
			\label{tab:components}
		\end{center}
	%\end{adjustbox}
	\end{table}
	\subsection{Breadboard}
	The breadboard can be divided into 5 segments.  In each of the green segements, the pins are internally connected so as to have the same voltage.  Similarly, in the central segments, the pins in each column  are internally connected in the same fashion as the blue columns. 
	\subsection{Seven Segment Display}
	The seven segment display in Fig. \ref{fig:sevenseg} has eight pins, $a, b, c, d, e, f, g$ and $dot$ that take an active LOW input, i.e.  the LED will glow only if the input is connected to ground.  Each of these pins is connected to an LED segment.  The $dot$ pin is  reserved for the $\cdot$ LED.  
	
	\subsection{Arduino}
	The Arduino Uno has some ground pins, analog input pins A0-A3 and digital pins D1-D13 that can be used for both input as well as output. It also has two power pins that can generate 3.3$V$ and 5$V$.  In the following exercises, only the GND, 5$V$ and digital pins will be used.
	
	
	\section{Display Control through Hardware }
	\subsection{Powering the Display}
	
%	\begin{problem}
	\textbf{Problem 2.1.} Plug the display to the breadboard in Fig. \ref{fig:breadboard} and make the connections in Table \ref{tab:connections}.  Henceforth, all 5V and GND connections will be made from the breadboard.
	%\end{problem}
	\begin{table}[h!]
	\begin{center}
		\begin{tabular}{ |c|c| } 
			\hline
			\textbf{Arduino} & \textbf{Breadboard}  \\ 
			\hline
			5$V$ & Top Green  \\ 
			\hline
			GND & Bottom Green   \\ 
			\hline
		\end{tabular}
		\caption{}
		\label{tab:connections}
	\end{center}
\end{table}
	\begin{figure}[!h]
		\begin{center}
			\includegraphics[width=\columnwidth]{./figs/breadboard}
		\end{center}
		\caption{}
		\label{fig:breadboard}
	\end{figure}
	
	
	%
	%\begin{center}
	%\includegraphics[scale=1]{sevenseg}
	%\end{center}
	
	
	\textbf{Problem 2.2.} Make the  connections in Table \ref{table:ard_disp}.  
	
		\begin{table}[h!]
	\begin{center}
			\begin{tabular}{ |c|c|c| } 
				\hline
				\textbf{Breadboard} &  & \textbf{Display}  \\ 
				\hline
				5$V$ & \text{Resistor} & \text{COM}  \\ 
				\hline
				\text{GND} &  & \text{DOT}  \\ 
				\hline
			\end{tabular}
			\caption{}
			\label{table:ard_disp}
	\end{center}
	\end{table}

	\begin{figure}[!h]
		\begin{center}
		\includegraphics[width=6cm,height=9cm]{./sevenseg}
		\end{center}
		\caption{}
		\label{fig:sevenseg}
	\end{figure}

		\textbf{Problem 2.3.} Connect the Arduino to the computer. The DOT led should glow.
	%\end{problem}
	\subsection{Controlling the Display}
	Fig. \ref{fig:sevenseg12} explains how to get decimal digits using the seven segment display. GND=0.  \\
%	\begin{problem}
		\textbf{Problem 2.4.} Generate the number 1 on the display by connecting only the pins $b$ and $c$ to GND (=0). This corresponds to the  first row of \ref{table:arduioport}. 1 means not connecting to GND.\\
	%\end{problem}	
%	\begin{problem}
		\textbf{Problem 2.5.} Repeat the above exercise to generate the number 2 on the display.\\
	%\end{problem}	
	%
	%\begin{problem}
		\textbf{Problem 2.6.} Draw the numbers 0-9 as in Fig. \ref{fig:sevenseg12} and complete Table \ref{table:arduioport}
	%\end{problem}	
%	\input{./figs/arduinoport}
		\begin{table}[h!]
		\begin{center}
			\begin{tabular}{ |c|c|c|c|c|c|c|c| } 
				\hline
				a & b & c & d & e & f & g & decimal \\ 
				\hline
				0 & 0 & 0 & 0 & 0 & 0 & 1 & 0 \\ 
				\hline
				1 & 0 & 0 & 1 & 1 & 1 & 1 & 1 \\ 
				\hline
				0 & 0 & 1 & 0 & 0 & 1 & 0 & 2 \\ 
				\hline
				0 & 0 & 0 & 0 & 1 & 1 & 0 & 3 \\
				\hline
				0 & 1 & 0 & 0 & 1 & 0 & 0 & 4 \\
				\hline
				1 & 0 & 0 & 1 & 1 & 0 & 0 & 5 \\
				\hline
				0 & 1 & 0 & 0 & 1 & 0 & 0 & 6 \\ 
				\hline
				0 & 0 & 0 & 1 & 1 & 1 & 1 & 7 \\
				\hline
				0 & 0 & 0 & 0 & 0 & 0 & 0 & 8 \\
				\hline
				0 & 0 & 0 & 0 & 1 & 0 & 0 & 9 \\ 
				\hline
			\end{tabular}
			\caption{}
			\label{table:arduioport}
		\end{center}
	\end{table}
	%
	\begin{figure}[!h]
		\begin{center}
		\includegraphics[width=7cm,height=4cm]{./dis12}
		\end{center}
		\caption{}
		\label{fig:sevenseg12}
	\end{figure}
	%
	\section{Display Control through Software}
	%\begin{problem}
		\textbf{Problem 3.1.} Make connections according to Table \ref{table:ard_disp_num} \\
		\begin{table}[h!]
		\begin{center}
			\begin{tabular}{ |c|c|c|c|c|c|c|c| } 
				\hline
				\textbf{Arduino} & 2 & 3 & 4 & 5 & 6 & 7 & 8  \\ 
				\hline
				\textbf{Display} & a & b & c & d & e & f & g  \\ 
				\hline
			\end{tabular}
			\caption{}
			\label{table:ard_disp_num}
		\end{center}
	\end{table}
	%\end{problem}
	%\begin{problem}
		\textbf{Problem 3.2.} Download the following code using the arduino IDE and execute \\
	\begin{figure}[h!]
	\begin{center}
		\includegraphics[width=7cm,height=2cm]{./code}
	\end{center}
	
\end{figure} 	%\end{problem}
	%\begin{problem}
		\textbf{Problem 3.3.} Now generate the numbers 0-9 by modifying the above program.
	%\end{problem}
	

	
\end{document}