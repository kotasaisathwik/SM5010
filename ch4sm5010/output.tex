\documentclass[journal,12pt,twocolumn]{IEEEtran}
%
\usepackage{setspace}
\usepackage{gensymb}
\usepackage{xcolor}
\usepackage{caption}
\usepackage[hyphens,spaces,obeyspaces]{url}
%\usepackage{subcaption}
%\doublespacing
\singlespacing

%\usepackage{graphicx}
%\usepackage{amssymb}
%\usepackage{relsize}
\usepackage[cmex10]{amsmath}
\usepackage{mathtools}
%\usepackage{amsthm}
%\interdisplaylinepenalty=2500
%\savesymbol{iint}
%\usepackage{txfonts}
%\restoresymbol{TXF}{iint}
%\usepackage{wasysym}
\usepackage{amsthm}
\usepackage{mathrsfs}
\usepackage{txfonts}
\usepackage{stfloats}
\usepackage{cite}
\usepackage{cases}
\usepackage{subfig}
%\usepackage{xtab}
\usepackage{longtable}
\usepackage{multirow}
%\usepackage{algorithm}
%\usepackage{algpseudocode}
\usepackage{enumerate}
\usepackage{mathtools}
%\usepackage{eenrc}
%\usepackage[framemethod=tikz]{mdframed}
\usepackage[breaklinks]{hyperref}
%\usepackage{breakcites}
\usepackage{listings}
\usepackage[latin1]{inputenc}                                 %%
\usepackage{color}                                            %%
\usepackage{array}                                            %%
\usepackage{longtable}                                        %%
\usepackage{calc}                                             %%
\usepackage{multirow}                                         %%
\usepackage{hhline}                                           %%
\usepackage{ifthen}                                           %%
%optionally (for landscape tables embedded in another document): %%
\usepackage{lscape}     


\usepackage{tikz}
\usepackage{circuitikz}
\usepackage{karnaugh-map}
\usepackage{pgf}
\usepackage[hyphenbreaks]{breakurl}

%\usepackage{url}
%\def\UrlBreaks{\do\/\do-}





%\usepackage{stmaryrd}


%\usepackage{wasysym}
%\newcounter{MYtempeqncnt}
\DeclareMathOperator*{\Res}{Res}
%\renewcommand{\baselinestretch}{2}
\renewcommand\thesection{\arabic{section}}
\renewcommand\thesubsection{\thesection.\arabic{subsection}}
\renewcommand\thesubsubsection{\thesubsection.\arabic{subsubsection}}

\renewcommand\thesectiondis{\arabic{section}}
\renewcommand\thesubsectiondis{\thesectiondis.\arabic{subsection}}
\renewcommand\thesubsubsectiondis{\thesubsectiondis.\arabic{subsubsection}}

% correct bad hyphenation here
\hyphenation{op-tical net-works semi-conduc-tor}

%\lstset{
	%language=C,
	%frame=single, 
	%breaklines=true
	%}

%\lstset{
	%%basicstyle=\small\ttfamily\bfseries,
	%%numberstyle=\small\ttfamily,
	%language=Octave,
	%backgroundcolor=\color{white},
	%%frame=single,
	%%keywordstyle=\bfseries,
	%%breaklines=true,
	%%showstringspaces=false,
	%%xleftmargin=-10mm,
	%%aboveskip=-1mm,
	%%belowskip=0mm
	%}

%\surroundwithmdframed[width=\columnwidth]{lstlisting}
\def\inputGnumericTable{}                                 %%
\lstset{
	%language=C,
	frame=single, 
	breaklines=true,
	columns=fullflexible
}

\graphicspath{ {./title/} } 
\begin{document}
		\bibliographystyle{IEEEtran}
	
	\title{
		
		
		\includegraphics[width=15cm, height=4cm]{title}
		\centering
		
	}
	\author{G V V Sharma$^{*}$% <-this % stops a space
		\thanks{*The author is with the Department
			of Electrical Engineering, Indian Institute of Technology, Hyderabad
			502285 India e-mail:  gadepall@iith.ac.in. All content in this manual is released under GNU GPL.  Free and open source.}}


% make the title area
\maketitle

\tableofcontents

\bigskip

\renewcommand{\thefigure}{\theenumi}
\renewcommand{\thetable}{\theenumi}

\begin{abstract}
%\boldmath
This manual explains Karnaugh maps (K-map) using don't care conditions.
\end{abstract}
%
\section{Don't Care Conditions}
\begin{enumerate}[1.]
\item {Don't Care Conditions: }
4 binary digits are used in the incrementing decoder \cite{gvv_kmap}.  However, only the numbers from 0-9 are used as input/output
in the decoder and we {\em don't care} about the numbers from 10-15.  This phenomenon can be addressed by revising the truth table in \cite{gvv_kmap}
to obtain Table \ref{table:dont_care_table}.
\begin{table}[h!]
	\begin{center}
		\begin{tabular}{ |c|c|c|c|c|c|c|c| } 
			\hline
			Z & Y & X & W & \textbf{D} & \textbf{C} & \textbf{B} & \textbf{A}  \\ 
			\hline
			0 & 0 & 0 & 0 & 0 & 0 & 0 & 1  \\ 
			\hline
			0 & 0 & 0 & 1 & 0 & 0 & 1 & 0  \\ 
			\hline
			0 & 0 & 1 & 0 & 0 & 0 & 1 & 1  \\ 
			\hline
			0 & 0 & 1 & 1 & 0 & 1 & 0 & 0  \\
			\hline
			0 & 1 & 0 & 0 & 0 & 1 & 0 & 1  \\
			\hline
			0 & 1 & 0 & 1 & 0 & 1 & 1 & 0  \\
			\hline
			0 & 1 & 1 & 0 & 0 & 1 & 1 & 1  \\ 
			\hline
			0 & 1 & 1 & 1 & 1 & 0 & 0 & 0  \\
			\hline
			1 & 0 & 0 & 0 & 1 & 0 & 0 & 1  \\
			\hline
			1 & 0 & 0 & 1 & 0 & 0 & 0 & 0  \\ 
			\hline
				1 & 0 & 1 & 0 & - & - & - & -  \\ 
			\hline
				1 & 0 & 1 & 1 & - & - & - & -  \\ 
			\hline
				1 & 1 & 0 & 0 & - & - & - & -  \\ 
			\hline
				1 & 1 & 0 & 1 & - & - & - & -  \\ 
			\hline
				1 & 1 & 1 & 0 & - & - & - & -  \\ 
			\hline
				1 & 1 & 1 & 1 & - & - & - & -  \\ 
			\hline
		\end{tabular}
\caption{}
\label{table:dont_care_table}
\end{center}
\end{table}
\item  The revised K-map for A is available in Fig. \ref{fig:kmap_A_x}.  Show that 
\begin{equation}
	A = {W}^{\prime}
\end{equation}
\begin{figure}[!h]
	\resizebox {\columnwidth} {!} {
		\input{./figs/kmap_A_x}
	}
	\caption{K-map for $A$ with don't cares.}
	\label{fig:kmap_A_x}
\end{figure}
%
\item  The revised K-map for B is available in Fig. \ref{fig:kmap_B_x}.  Show that 
%
\begin{equation}
	B = WX^{\prime}Z^{\prime} + W^{\prime}X
\end{equation}
\begin{figure}[!h]
	\resizebox {\columnwidth} {!} {
		\input{./figs/kmap_B_x}
	}
	\caption{K-map for $B$ with don't cares.}
	\label{fig:kmap_B_x}
\end{figure}
%
\item  The revised K-map for C is available in Fig. \ref{fig:kmap_C_x}.  Show that 
%
\begin{equation}
	C = {X}^{\prime}{Y} + {W}^{\prime}{Y} + {W}{X}{Y}^{\prime}
\end{equation}
\begin{figure}[!h]
	\resizebox {\columnwidth} {!} {
		\input{./figs/kmap_C_x}
	}
	\caption{K-map for $C$ with don't cares.}
	\label{fig:kmap_C_x}
\end{figure}
%
\item  The revised K-map for D is available in Fig. \ref{fig:kmap_D_x}.  Show that 
%
\begin{equation}
	D  = {W}^{\prime}{Z} + {W}{X}{Y}
\end{equation}
\begin{figure}[!h]
	\resizebox {\columnwidth} {!} {
		\input{./figs/kmap_D_x}
	}
	\caption{K-map for $D$ with don't cares.}
	\label{fig:kmap_D_x}
\end{figure}

\item Verify the incrementing decoder with don't care conditions using the arduino.
\item {Display Decoder:}
Use K-maps to obtain the minimized expressions for $a,b,c,d,e,f,g$ in terms of $A,B,C,D$ with  don't care conditions.\\
\textbf{Solution:}\\
With Don't Care:\\
from Fig. \ref{fig:kmap_a}
\begin{equation}
	a = CB^{\prime}A^{\prime}+D^{\prime}C^{\prime}B^{\prime}A
\end{equation}
from Fig. \ref{fig:kmap_b}
\begin{equation}
	b = CB^{\prime}A+CBA^{\prime}
\end{equation}
from Fig.  \ref{fig:kmap_c}
\begin{equation}
	c = C^{\prime}BA^{\prime}
\end{equation}
from Fig. \ref{fig:kmap_d}
\begin{equation}
	d = CB^{\prime}A^{\prime}+CBA+C^{\prime}B^{\prime}A
\end{equation}
from Fig. \ref{fig:kmap_e}
\begin{equation}
	e = A+CB^{\prime}
\end{equation}
from Fig. \ref{fig:kmap_f}
\begin{equation}
	f = BA+D^{\prime}C^{\prime}A+C^{\prime}B
\end{equation}
from Fig. \ref{fig:kmap_g}
\begin{equation}
	g = D^{\prime}C^{\prime}B^{\prime}+CBA
\end{equation}
\begin{figure}[!h]
	\resizebox {\columnwidth} {!} {
		\input{./figs/a_kmap}
	}
	\caption{K-map for $a$ with don't cares.}
	\label{fig:kmap_a}
\end{figure}
\begin{figure}[!h]
	\resizebox {\columnwidth} {!} {
		\begin{karnaugh-map}[4][4][1][][]
    \maxterms{0,1,2,3,4,7,8,9,10,11,12,13,14,15}
    \minterms{5,6}
    \implicant{5}{5}
    \implicant{6}{6}
    % note: posistion for start of \draw is (0, Y) where Y is
    % the Y size(number of cells high) in this case Y=2
    \draw[color=black, ultra thin] (0, 4) --
    node [pos=0.7, above right, anchor=south west] {$BA$} % Y label
    node [pos=0.7, below left, anchor=north east] {$DC$} % X label
    ++(135:1);
        
    \end{karnaugh-map}

	}
	\caption{K-map for $b$ with don't cares.}
	\label{fig:kmap_b}
\end{figure}
\begin{figure}[!h]
	\resizebox {\columnwidth} {!} {
		\begin{karnaugh-map}[4][4][1][][]
   \maxterms{0,1,3,4,5,6,7,8,9}
    \minterms{2}
    \implicantedge{2}{2}{10}{10}
	\indeterminants{10,11,12,13,14,15}        
    \draw[color=black, ultra thin] (0, 4) --
    node [pos=0.7, above right, anchor=south west] {$BA$} % Y label
    node [pos=0.7, below left, anchor=north east] {$DC$} % X label
    ++(135:1);
        
    \end{karnaugh-map}

	}
	\caption{K-map for $c$ with don't cares.}
	\label{fig:kmap_c}
\end{figure}
\begin{figure}[!h]
	\resizebox {\columnwidth} {!} {
		\begin{karnaugh-map}[4][4][1][][]
    \maxterms{1,2,3,4,5,6,7,8,10,11,12,13,14,15}
    \minterms{0,9}
    \implicant{0}{0}
    \implicant{9}{9}
    %\implicantedge{1}{1}{9}{9}
    % note: posistion for start of \draw is (0, Y) where Y is
    % the Y size(number of cells high) in this case Y=2
    \draw[color=black, ultra thin] (0, 4) --
    node [pos=0.7, above right, anchor=south west] {$XW$} % Y label
    node [pos=0.7, below left, anchor=north east] {$ZY$} % X label
    ++(135:1);
        
    \end{karnaugh-map}

	}
	\caption{K-map for $d$ with don't cares.}
	\label{fig:kmap_d}
\end{figure}
\begin{figure}[!h]
	\resizebox {\columnwidth} {!} {
		\begin{karnaugh-map}[4][4][1][][]
    \maxterms{0,2,6,8,10,11,12,13,14,15}
    \minterms{1,3,4,5,7,9}
    \implicantedge{1}{1}{9}{9}
    \implicant{4}{5}
    \implicant{1}{7}
    % note: posistion for start of \draw is (0, Y) where Y is
    % the Y size(number of cells high) in this case Y=2
    \draw[color=black, ultra thin] (0, 4) --
    node [pos=0.7, above right, anchor=south west] {$BA$} % Y label
    node [pos=0.7, below left, anchor=north east] {$DC$} % X label
    ++(135:1);
        
    \end{karnaugh-map}

	}
	\caption{K-map for $e$ with don't cares.}
	\label{fig:kmap_e}
\end{figure}
\begin{figure}[!h]
	\resizebox {\columnwidth} {!} {
		\begin{karnaugh-map}[4][4][1][][]
    \maxterms{0,4,5,6,8,9,10,11,12,13,14,15}
    \minterms{1,2,3,7}
    \implicant{1}{3}
    \implicant{3}{2}
    \implicant{3}{7}
    % note: posistion for start of \draw is (0, Y) where Y is
    % the Y size(number of cells high) in this case Y=2
    \draw[color=black, ultra thin] (0, 4) --
    node [pos=0.7, above right, anchor=south west] {$BA$} % Y label
    node [pos=0.7, below left, anchor=north east] {$DC$} % X label
    ++(135:1);
        
    \end{karnaugh-map}

	}
	\caption{K-map for $f$ with don't cares.}
	\label{fig:kmap_f}
\end{figure}
\begin{figure}[!h]
	\resizebox {\columnwidth} {!} {
		\begin{karnaugh-map}[4][4][1][][]
    \maxterms{2,3,4,5,6,8,9,10,11,12,13,14,15}
    \minterms{0,1,7}
    \implicant{0}{1}
    \implicant{7}{7}
    % note: posistion for start of \draw is (0, Y) where Y is
    % the Y size(number of cells high) in this case Y=2
    \draw[color=black, ultra thin] (0, 4) --
    node [pos=0.7, above right, anchor=south west] {$BA$} % Y label
    node [pos=0.7, below left, anchor=north east] {$DC$} % X label
    ++(135:1);
        
    \end{karnaugh-map}

	}
	\caption{K-map for $g$ with don't cares.}
	\label{fig:kmap_g}
\end{figure}
\item Verify the display decoder with don't care conditions using arduino.
\end{enumerate}

\bibliography{IEEEabrv,gvv_dontcare}


\end{document}


