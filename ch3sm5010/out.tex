\documentclass[journal,12pt,twocolumn]{IEEEtran}
%
\usepackage{setspace}
\usepackage{gensymb}
\usepackage{xcolor}
\usepackage{caption}
\usepackage[hyphens,spaces,obeyspaces]{url}
%\usepackage{subcaption}
%\doublespacing
\singlespacing

%\usepackage{graphicx}
%\usepackage{amssymb}
%\usepackage{relsize}
\usepackage[cmex10]{amsmath}
\usepackage{mathtools}
%\usepackage{amsthm}
%\interdisplaylinepenalty=2500
%\savesymbol{iint}
%\usepackage{txfonts}
%\restoresymbol{TXF}{iint}
%\usepackage{wasysym}
\usepackage{amsthm}
\usepackage{mathrsfs}
\usepackage{txfonts}
\usepackage{stfloats}
\usepackage{cite}
\usepackage{cases}
\usepackage{subfig}
%\usepackage{xtab}
\usepackage{longtable}
\usepackage{multirow}
%\usepackage{algorithm}
%\usepackage{algpseudocode}
\usepackage{enumerate}
\usepackage{mathtools}
%\usepackage{eenrc}
%\usepackage[framemethod=tikz]{mdframed}
\usepackage[breaklinks]{hyperref}
%\usepackage{breakcites}
\usepackage{listings}
\usepackage[latin1]{inputenc}                                 %%
\usepackage{color}                                            %%
\usepackage{array}                                            %%
\usepackage{longtable}                                        %%
\usepackage{calc}                                             %%
\usepackage{multirow}                                         %%
\usepackage{hhline}                                           %%
\usepackage{ifthen}                                           %%
%optionally (for landscape tables embedded in another document): %%
\usepackage{lscape}     
\usepackage{amsmath}
\usepackage{tikz}
\usepackage{circuitikz}
\usepackage{karnaugh-map}
\usepackage{pgf}
\usepackage[hyphenbreaks]{breakurl}
\usepackage{enumitem}

%\usepackage{url}
%\def\UrlBreaks{\do\/\do-}





%\usepackage{stmaryrd}


%\usepackage{wasysym}
%\newcounter{MYtempeqncnt}
\DeclareMathOperator*{\Res}{Res}
%\renewcommand{\baselinestretch}{2}
\renewcommand\thesection{\arabic{section}}
\renewcommand\thesubsection{\thesection.\arabic{subsection}}
\renewcommand\thesubsubsection{\thesubsection.\arabic{subsubsection}}

\renewcommand\thesectiondis{\arabic{section}}
\renewcommand\thesubsectiondis{\thesectiondis.\arabic{subsection}}
\renewcommand\thesubsubsectiondis{\thesubsectiondis.\arabic{subsubsection}}

% correct bad hyphenation here
\hyphenation{op-tical net-works semi-conduc-tor}

%\lstset{
	%language=C,
	%frame=single, 
	%breaklines=true
	%}

%\lstset{
	%%basicstyle=\small\ttfamily\bfseries,
	%%numberstyle=\small\ttfamily,
	%language=Octave,
	%backgroundcolor=\color{white},
	%%frame=single,
	%%keywordstyle=\bfseries,
	%%breaklines=true,
	%%showstringspaces=false,
	%%xleftmargin=-10mm,
	%%aboveskip=-1mm,
	%%belowskip=0mm
	%}

%\surroundwithmdframed[width=\columnwidth]{lstlisting}
\def\inputGnumericTable{}                                 %%
\lstset{
	%language=C,
	frame=single, 
	breaklines=true,
	columns=fullflexible
}

\graphicspath{ {./title/} } 
\begin{document}
	
	
	\title{
		
		
		\includegraphics[width=15cm, height=4cm]{title}
		\centering
		
	}
	\author{G V V Sharma$^{*}$% <-this % stops a space
		\thanks{*The author is with the Department
			of Electrical Engineering, Indian Institute of Technology, Hyderabad
			502285 India e-mail:  gadepall@iith.ac.in. All content in this manual is released under GNU GPL.  Free and open source.}}

% make the title area
\maketitle

\tableofcontents

\bigskip

\renewcommand{\thefigure}{\theenumi}
\renewcommand{\thetable}{\theenumi}

\begin{abstract}
%\boldmath
This manual explains Karnaugh maps (K-map) by finding the
logic functions for the incrementing decoder.
\end{abstract}

%
\section{Incrementing Decoder}
The incrementing decoder   takes the numbers $0,1,\dots,9$ in binary as inputs and generates
the consecutive number as output.  The corresponding truth table is available in Table. \ref{table:counter_decoder}.
\begin{table}[!h]
	\centering	
	\input{./figs/counter_decoder.tex}
	\caption{}
	\label{table:counter_decoder}
\end{table}
\section{Karnaugh Map}
Using Boolean logic, output $A$  in Table \ref{table:counter_decoder} can be expressed in terms of the inputs $W,X,Y,Z$ as
\begin{equation}
\label{eq:A}
A = W^{\prime}X^{\prime}Y^{\prime}Z^{\prime}+W^{\prime}XY^{\prime}Z^{\prime}\\
+W^{\prime}X^{\prime}YZ^{\prime}+W^{\prime}XYZ^{\prime}+W^{\prime}X^{\prime}Y^{\prime}Z
\end{equation}
\begin{enumerate}
\item K-Map for $A$: 
The expression in \eqref{eq:A}  can be minimized using the K-map in Fig. \ref{fig:kmap_A}.
In Fig. \ref{fig:kmap_A},  the {\em implicants} in boxes 0,2,4,6 result in $W^{\prime}Z^{\prime}$.  The implicants in
boxes 0,8 result in $W^{\prime}X^{\prime}Y^{\prime}$.  Thus, after minimization using Fig. \ref{eq:kmap_A},  \eqref{eq:A} can be expressed as
\begin{equation}
	\label{eq:kmap_A}
	A = W^{\prime}Z^{\prime}+W^{\prime}X^{\prime}Y^{\prime}
\end{equation}
%
Using the fact that
\begin{align}
	\label{eq:boolean}
	\begin{split}
		X+X^{\prime} &= 1
		\\
		XX^{\prime} &= 0,
	\end{split}
\end{align}
%
derive \eqref{eq:kmap_A} from \eqref{eq:A} algebraically.\\
\textbf{Solution: } 
\begin{equation}
	A = W^{\prime}X^{\prime}Y^{\prime}Z^{\prime}+W^{\prime}XY^{\prime}Z^{\prime}+W^{\prime}X^{\prime}YZ^{\prime}+W^{\prime}XYZ^{\prime}+W^{\prime}X^{\prime}Y^{\prime}Z 
\end{equation}
\begin{equation}
	A = W^{\prime}X^{\prime}Y^{\prime}+W^{\prime}X^{\prime}Z^{\prime}
	+W^{\prime}Y^{\prime}Z^{\prime}+W^{\prime}XYZ^{\prime}\\
\end{equation}
\begin{equation}
	A = W^{\prime}X^{\prime}Y^{\prime}+W^{\prime}Z^{\prime}(X^{\prime}+Y^{\prime}+XY)\\
\end{equation}
\begin{equation}
	A = W^{\prime}X^{\prime}Y^{\prime}+W^{\prime}Z^{\prime}((XY)^{\prime}+XY)\\
\end{equation}
\begin{equation}
		A = W^{\prime}Z^{\prime}+W^{\prime}X^{\prime}Y^{\prime}	
\end{equation}

%
%
%
\begin{figure}[!h]
	\resizebox {\columnwidth} {!} {
		\input{./figs/kmap_A}
	}
	\caption{K-map for $A$.}
	\label{fig:kmap_A}
\end{figure}
%
\item K-Map for $B$:
From Table \ref{table:counter_decoder}, using boolean logic,
\begin{equation}
	\label{eq:B}
	B = WX^{\prime}Y^{\prime}Z^{\prime} + W^{\prime}XY^{\prime}Z^{\prime}
	+WX^{\prime}YZ^{\prime}
	+W^{\prime}XYZ^{\prime}
\end{equation}
%
\begin{figure}[!h]
	\resizebox {\columnwidth} {!} {
		\input{./figs/kmap_B}
	}
	\caption{K-map for $B$.}
	\label{fig:kmap_B}
\end{figure}
%
Show that \eqref{eq:B} can be reduced to
\begin{equation}
	\label{eq:kmap_B}
	B = WX^{\prime}Z^{\prime} + W^{\prime}XZ^{\prime}
\end{equation}
using Fig. \ref{fig:kmap_B}.
\item Derive \eqref{eq:kmap_B} from \eqref{eq:B} algebraically using \eqref{eq:boolean}.
\textbf{Solution:}
\begin{equation}
	\label{eq:B}
	B = WX^{\prime}Y^{\prime}Z^{\prime} + W^{\prime}XY^{\prime}Z^{\prime}
	+WX^{\prime}YZ^{\prime}
	+W^{\prime}XYZ^{\prime}
\end{equation}
\begin{equation}
	B = WX^{\prime}Z^{\prime} + W^{\prime}XZ^{\prime}
\end{equation}
%
\item {K-Map for $C$: }
From Table \ref{table:counter_decoder}, using boolean logic,
\begin{equation}
	\label{eq:C}
	C = WXY^{\prime}Z^{\prime} + W^{\prime}X^{\prime}YZ^{\prime}
	+WX^{\prime}YZ^{\prime}
	+W^{\prime}XYZ^{\prime}
\end{equation}
%
%
\begin{figure}[!h]
	\resizebox {\columnwidth} {!} {
		\input{./figs/kmap_C}
	}
	\caption{K-map for $C$.}
	\label{fig:kmap_C}
\end{figure}
%
Show that \eqref{eq:C} can be reduced to
\begin{equation}
	\label{eq:kmap_C}
	C = WXY^{\prime}Z^{\prime}  +  X^{\prime}YZ^{\prime} + W^{\prime}YZ^{\prime}
\end{equation}
using Fig. \ref{fig:kmap_C}.
%
\item 
Derive \eqref{eq:kmap_C} from \eqref{eq:C} algebraically using \eqref{eq:boolean}.
\textbf{Solution:}
\begin{equation}
	C = WXY^{\prime}Z^{\prime} + W^{\prime}X^{\prime}YZ^{\prime}
	+WX^{\prime}YZ^{\prime}
	+W^{\prime}XYZ^{\prime}
\end{equation}
\begin{equation}
	C = WXY^{\prime}Z^{\prime}  +  X^{\prime}YZ^{\prime} + W^{\prime}YZ^{\prime}
\end{equation}
%
\item {K-Map for $D$: }
From Table \ref{table:counter_decoder}, using boolean logic,
\begin{equation}
	\label{eq:D}
	D = WXYZ^{\prime} + W^{\prime}X^{\prime}Y^{\prime}Z
\end{equation}
%
\begin{figure}[!h]
	\resizebox {\columnwidth} {!} {
		\input{./figs/kmap_D}
	}
	\caption{K-map for $D$.}
	\label{fig:kmap_D}
\end{figure}
%
\item 
Minimize \eqref{eq:D} using Fig. \ref{fig:kmap_D}.
%
\item Download the code in
\begin{lstlisting}
	wget https://raw.githubusercontent.com/gadepall/arduino/master/7447/codes/inc_dec/inc_dec.ino
\end{lstlisting}
%
and modify it using the K-Map equations for A,B,C and D. Execute and verify.
\item {Display Decoder:}
Table \ref{table:disp_dec} is the truth table for the display decoder.  
Use K-maps to obtain the minimized expressions for $a,b,c,d,e,f,g$ in terms of $A,B,C,D$ without don't care conditions.
%

	\begin{table}[h!]
	\begin{center}
		\begin{tabular}{ |c|c|c|c|c|c|c|c|c|c|c|c| } 
			\hline
			D & C & B & A & \textbf{a} & \textbf{b} & \textbf{c} & \textbf{d} & \textbf{e} & \textbf{f} & \textbf{g} & Decimal \\ 
			\hline
			0 & 0 & 0 & 0 & 0 & 0 & 0 & 0 & 0 & 0 & 1 & 0 \\ 
			\hline
			0 & 0 & 0 & 1 & 1 & 0 & 0 & 1 & 1 & 1 & 1 & 1 \\ 
			\hline
			0 & 0 & 1 & 0 & 0 & 0 & 1 & 0 & 0 & 1 & 0 & 2 \\ 
			\hline
			0 & 0 & 1 & 1 & 0 & 0 & 0 & 0 & 1 & 1 & 0 & 3 \\
			\hline
			0 & 1 & 0 & 0 & 0 & 1 & 0 & 0 & 1 & 0 & 0 & 4 \\
			\hline
			0 & 1 & 0 & 1 & 1 & 0 & 0 & 1 & 1 & 0 & 0 & 5 \\
			\hline
			0 & 1 & 1 & 0 & 0 & 1 & 0 & 0 & 1 & 0 & 0 & 6 \\ 
			\hline
			0 & 1 & 1 & 1 & 0 & 0 & 0 & 1 & 1 & 1 & 1 & 7 \\
			\hline
			1 & 0 & 0 & 0 & 0 & 0 & 0 & 0 & 0 & 0 & 0 & 8 \\
			\hline
			1 & 0 & 0 & 1 & 0 & 0 & 0 & 0 & 1 & 0 & 0 & 9 \\ 
			\hline
		\end{tabular}
		\caption{Truth table for display decoder.}
		\label{table:disp_dec}
	\end{center}
\end{table}
\end{enumerate}
\textbf{Solution:}\\
Without DON'T CARE: \\
from Fig. \ref{fig:kmap_a}
\begin{equation}
	a = D^{\prime}C^{\prime}B^{\prime}A+D^{\prime}CB^{\prime}A^{\prime}
\end{equation}
from Fig. \ref{fig:kmap_b}
\begin{equation}
	b = D^{\prime}CB^{\prime}A+D^{\prime}CBA^{\prime}
\end{equation}
from Fig.  \ref{fig:kmap_c}
\begin{equation}
	c = D^{\prime}C^{\prime}BA^{\prime}
\end{equation}
from Fig. \ref{fig:kmap_d}
\begin{equation}
	d = D^{\prime}CB^{\prime}A^{\prime}+D^{\prime}CBA+C^{\prime}B^{\prime}A
\end{equation}
from Fig. \ref{fig:kmap_e}
\begin{equation}
	e = D^{\prime}A+C^{\prime}B^{\prime}A+D^{\prime}CB^{\prime}
\end{equation}
from Fig. \ref{fig:kmap_f}
\begin{equation}
	f = D^{\prime}BA+D^{\prime}C^{\prime}A+D^{\prime}C^{\prime}B
\end{equation}
from Fig. \ref{fig:kmap_g}
\begin{equation}
	g = D^{\prime}C^{\prime}B^{\prime}+D^{\prime}CBA
\end{equation}
\begin{figure}[!h]
	\resizebox {\columnwidth} {!} {
		\input{./figs/a_kmap}
	}
	\caption{K-map for $a$.}
	\label{fig:kmap_a}
\end{figure}
\begin{figure}[!h]
	\resizebox {\columnwidth} {!} {
		\begin{karnaugh-map}[4][4][1][][]
    \maxterms{0,1,2,3,4,7,8,9,10,11,12,13,14,15}
    \minterms{5,6}
    \implicant{5}{5}
    \implicant{6}{6}
    % note: posistion for start of \draw is (0, Y) where Y is
    % the Y size(number of cells high) in this case Y=2
    \draw[color=black, ultra thin] (0, 4) --
    node [pos=0.7, above right, anchor=south west] {$BA$} % Y label
    node [pos=0.7, below left, anchor=north east] {$DC$} % X label
    ++(135:1);
        
    \end{karnaugh-map}

	}
	\caption{K-map for $b$.}
	\label{fig:kmap_b}
\end{figure}
\begin{figure}[!h]
	\resizebox {\columnwidth} {!} {
		\begin{karnaugh-map}[4][4][1][][]
   \maxterms{0,1,3,4,5,6,7,8,9}
    \minterms{2}
    \implicantedge{2}{2}{10}{10}
	\indeterminants{10,11,12,13,14,15}        
    \draw[color=black, ultra thin] (0, 4) --
    node [pos=0.7, above right, anchor=south west] {$BA$} % Y label
    node [pos=0.7, below left, anchor=north east] {$DC$} % X label
    ++(135:1);
        
    \end{karnaugh-map}

	}
	\caption{K-map for $c$.}
	\label{fig:kmap_c}
\end{figure}
\begin{figure}[!h]
	\resizebox {\columnwidth} {!} {
		\begin{karnaugh-map}[4][4][1][][]
    \maxterms{1,2,3,4,5,6,7,8,10,11,12,13,14,15}
    \minterms{0,9}
    \implicant{0}{0}
    \implicant{9}{9}
    %\implicantedge{1}{1}{9}{9}
    % note: posistion for start of \draw is (0, Y) where Y is
    % the Y size(number of cells high) in this case Y=2
    \draw[color=black, ultra thin] (0, 4) --
    node [pos=0.7, above right, anchor=south west] {$XW$} % Y label
    node [pos=0.7, below left, anchor=north east] {$ZY$} % X label
    ++(135:1);
        
    \end{karnaugh-map}

	}
	\caption{K-map for $d$.}
	\label{fig:kmap_d}
\end{figure}
\begin{figure}[!h]
	\resizebox {\columnwidth} {!} {
		\begin{karnaugh-map}[4][4][1][][]
    \maxterms{0,2,6,8,10,11,12,13,14,15}
    \minterms{1,3,4,5,7,9}
    \implicantedge{1}{1}{9}{9}
    \implicant{4}{5}
    \implicant{1}{7}
    % note: posistion for start of \draw is (0, Y) where Y is
    % the Y size(number of cells high) in this case Y=2
    \draw[color=black, ultra thin] (0, 4) --
    node [pos=0.7, above right, anchor=south west] {$BA$} % Y label
    node [pos=0.7, below left, anchor=north east] {$DC$} % X label
    ++(135:1);
        
    \end{karnaugh-map}

	}
	\caption{K-map for $e$.}
	\label{fig:kmap_e}
\end{figure}
\begin{figure}[!h]
	\resizebox {\columnwidth} {!} {
		\begin{karnaugh-map}[4][4][1][][]
    \maxterms{0,4,5,6,8,9,10,11,12,13,14,15}
    \minterms{1,2,3,7}
    \implicant{1}{3}
    \implicant{3}{2}
    \implicant{3}{7}
    % note: posistion for start of \draw is (0, Y) where Y is
    % the Y size(number of cells high) in this case Y=2
    \draw[color=black, ultra thin] (0, 4) --
    node [pos=0.7, above right, anchor=south west] {$BA$} % Y label
    node [pos=0.7, below left, anchor=north east] {$DC$} % X label
    ++(135:1);
        
    \end{karnaugh-map}

	}
	\caption{K-map for $f$.}
	\label{fig:kmap_f}
\end{figure}
\begin{figure}[!h]
	\resizebox {\columnwidth} {!} {
		\begin{karnaugh-map}[4][4][1][][]
    \maxterms{2,3,4,5,6,8,9,10,11,12,13,14,15}
    \minterms{0,1,7}
    \implicant{0}{1}
    \implicant{7}{7}
    % note: posistion for start of \draw is (0, Y) where Y is
    % the Y size(number of cells high) in this case Y=2
    \draw[color=black, ultra thin] (0, 4) --
    node [pos=0.7, above right, anchor=south west] {$BA$} % Y label
    node [pos=0.7, below left, anchor=north east] {$DC$} % X label
    ++(135:1);
        
    \end{karnaugh-map}

	}
	\caption{K-map for $g$.}
	\label{fig:kmap_g}
\end{figure}
\end{document}


